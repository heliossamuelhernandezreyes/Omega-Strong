\documentclass[11pt]{article}

\usepackage{amsmath,amssymb}
\usepackage{geometry}
\usepackage{graphicx}
\usepackage{hyperref}
\geometry{margin=2.5cm}

\title{Omega-Strong:\\
A Geometric Saturation Regime for Classical Strong-Field Gravity}

\author{Helios Samuel Hernández Reyes}
\date{\today}

\begin{document}
\maketitle

\begin{abstract}
We present \emph{Omega-Strong}, a classical effective strong-field regime of gravity
based on geometric saturation of curvature.
The model provides a proof-of-concept mechanism by which spacetime singularities
can be dynamically avoided without invoking exotic matter or quantum gravity.
A scalar degree of freedom induces a curvature-controlled saturation that
suppresses scalar dynamics at high curvature while recovering General Relativity
in low-curvature regimes.
Numerical solutions demonstrate the existence of regular strong-field cores
with finite curvature and a geometrically controlled transition scale.
\end{abstract}

\section{Motivation}

General Relativity predicts the formation of curvature singularities in
strong gravitational collapse, signaling a breakdown of the classical theory.
This motivates the exploration of effective mechanisms by which gravity may
self-regulate in regimes of extreme curvature.

Omega-Strong is motivated by the hypothesis that the same scalar degree of freedom
responsible for controlled deviations from General Relativity in weak-field
regimes may also regulate the strong-field regime through purely geometric effects.

\section{Conceptual Framework}

We introduce a curvature scale $R_\star$ such that the theory exhibits two
distinct regimes:
\begin{itemize}
\item For $R \gg R_\star$, the system enters a \emph{saturated regime} where
scalar dynamics are dynamically suppressed and curvature remains finite.
\item For $R \lesssim R_\star$, the theory smoothly transitions back to a
General-Relativity-like regime with standard scalar dynamics.
\end{itemize}

Crucially, the transition scale is controlled geometrically and does not depend
on fine-tuning of the scalar potential.

\section{Effective Model}

The model consists of a scalar field $\Omega$ coupled to gravity through an
effective curvature-dependent coupling function.
At high curvature, the effective coupling saturates, suppressing scalar gradients
and preventing the growth of curvature invariants.

The model is treated as a classical effective theory and is not intended as a
complete ultraviolet description.

\section{Numerical Solutions}

We solve the static, spherically symmetric field equations numerically.
The solutions exhibit:
\begin{itemize}
\item A regular strong-field core with finite curvature.
\item Dynamical suppression of scalar gradients in the high-curvature region.
\item Activation of non-trivial scalar profiles only beyond a curvature threshold.
\end{itemize}

A representative scalar profile is shown in Fig.~\ref{fig:omega_profile}.

\begin{figure}[h]
\centering
\includegraphics[width=0.85\linewidth]{../omega_profile_v1.png}
\caption{Representative scalar profile $\Omega(r)$ in the Omega-Strong regime,
showing a saturated core and a curvature-controlled transition.}
\label{fig:omega_profile}
\end{figure}

\section{Scope and Limitations}

Omega-Strong is a classical effective model.
It does not claim to describe quantum gravity, provide direct observational
predictions, or resolve all singularities in a fundamental sense.
Its purpose is to demonstrate the \emph{existence} of a consistent and regular
strong-field regime.

\section{Relation to the Omega Program}

Omega-Strong forms part of the broader Omega research program, including:
\begin{itemize}
\item Omega-CDM: late-time cosmological phenomenology,
\item Omega-Strong: classical strong-field regime (this work),
\item Omega-UV and Omega-Dark: ongoing and future extensions.
\end{itemize}

\section{Status}

This document corresponds to Omega-Strong v1.
The model is considered closed at this stage.

\end{document}
